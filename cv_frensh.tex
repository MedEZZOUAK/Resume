%!TEX TS-program = xelatex
%!TEX encoding = UTF-8 Unicode

\documentclass[12pt, a4paper]{awesome-cv}

%-------------------------------------------------------------------------------
% PAGE & FONT
%-------------------------------------------------------------------------------
\geometry{left=1.35cm, top=1.0cm, right=1.35cm, bottom=0.7cm, footskip=0.4cm}
\fontdir[fonts/]

%-------------------------------------------------------------------------------
% COLORS
%-------------------------------------------------------------------------------
\definecolor{awesome-gray}{HTML}{333333}
\definecolor{awesome-teal}{HTML}{1F7A8C}
\colorlet{awesome}{awesome-teal}

\definecolor{darktext}{HTML}{333333}
\definecolor{text}{HTML}{333333}
\definecolor{graytext}{HTML}{555555}
\definecolor{lighttext}{HTML}{999999}
\definecolor{sectiondivider}{HTML}{B0B0B0}

\setbool{acvSectionColorHighlight}{true}
\renewcommand{\acvHeaderSocialSep}{\quad\textbullet\quad}
\renewcommand{\acvHeaderAfterSocialSkip}{4mm}
\renewcommand{\acvSectionTopSkip}{2mm}
\renewcommand{\acvSectionContentTopSkip}{2mm}

%-------------------------------------------------------------------------------
% FONT SIZING
%-------------------------------------------------------------------------------
\makeatletter
\renewcommand*{\headerpositionstyle}[1]{{\fontsize{8.2pt}{1em}\bodyfont\scshape\color{awesome} #1}}
\renewcommand*{\headeraddressstyle}[1]{{\fontsize{8.5pt}{1em}\headerfont\itshape\color{lighttext} #1}}
\renewcommand*{\headersocialstyle}[1]{{\fontsize{7.2pt}{1em}\headerfont\color{text} #1}}
\renewcommand*{\sectionstyle}[1]{{\fontsize{16pt}{1em}\bodyfont\bfseries\color{text}\@sectioncolor #1}}
\renewcommand*{\paragraphstyle}{\fontsize{9.2pt}{1.12em}\bodyfontlight\upshape\color{text}}
\renewcommand*{\entrytitlestyle}[1]{{\fontsize{10.3pt}{1em}\bodyfont\bfseries\color{darktext} #1}}
\renewcommand*{\entrypositionstyle}[1]{{\fontsize{8.2pt}{1em}\bodyfont\scshape\color{graytext} #1}}
\renewcommand*{\entrydatestyle}[1]{{\fontsize{8.2pt}{1em}\bodyfontlight\slshape\color{graytext} #1}}
\renewcommand*{\entrylocationstyle}[1]{{\fontsize{9pt}{1em}\bodyfontlight\slshape\color{awesome} #1}}
\renewcommand*{\descriptionstyle}[1]{{\fontsize{9.2pt}{1.12em}\bodyfontlight\upshape\color{text} #1}}
\renewcommand*{\skilltypestyle}[1]{{\fontsize{9.8pt}{1em}\bodyfont\bfseries\color{darktext} #1}}
\renewcommand*{\skillsetstyle}[1]{{\fontsize{8.8pt}{1em}\bodyfontlight\color{text} #1}}
\makeatother

%-------------------------------------------------------------------------------
% SPACING
%-------------------------------------------------------------------------------
\usepackage{enumitem}
\setlist[itemize]{leftmargin=1.6em, itemsep=0pt, topsep=2pt}

\setlength{\parskip}{4pt}
\setlength{\parindent}{0pt}

%-------------------------------------------------------------------------------
% PERSONAL INFO
%-------------------------------------------------------------------------------
\photo[circle,edge,left]{profile_pic.jpg}

\name{Mohammed}{EZ-ZOUAK}
\position{Ingénieur Logiciel Junior}
\address{Casablanca, Maroc}

\mobile{(+212) 06-0203-7451}
\email{m.ezzouak@outlook.com}
\linkedin{mohammed-ez-zouak}
\homepage{mezzouak.tech}

\begin{document}
\makecvheader[C]

%-------------------------------------------------------------------------------
% PROFIL
%-------------------------------------------------------------------------------
\cvsection{Profil}

\begin{cvparagraph}
  Ingénieur logiciel junior avec une expérience en environnements industriels et ESN. Intervention sur des systèmes backend en production, la gestion de bases de données relationnelles et l’évolution d’applications existantes sous contraintes opérationnelles réelles. Exposé à des contextes multi-clients impliquant des enjeux fonctionnels, techniques et opérationnels.
  \end{cvparagraph}  

%-------------------------------------------------------------------------------
% EXPÉRIENCE
%-------------------------------------------------------------------------------
\cvsection{Expérience}

\begin{cventries}

\cventry
  {Ingénieur Logiciel Junior}
  {\textcolor{awesome}{ITSS Paris}}
  {Casablanca, Maroc}
  {Août 2025 -- Présent (CDI)}
  {
    \begin{cvitems}
      \item {Contribution au développement de la plateforme \textbf{IJConnect}, successeur de \textbf{IJStats}, avec maintien de la cohérence fonctionnelle et des données durant la phase de transition.}
      \item {Intervention sur le backend Django : implémentation de la logique métier, fonctionnalités basées sur des sockets et gestion des migrations de base de données.}
      \item {Analyse et résolution de problèmes de performance et de stabilité (requêtes SQL lentes, timeouts, saturation des pools de connexions) via l’optimisation des requêtes et l’indexation.}
      \item {Travail dans un environnement ESN~multi-clients pour des organisations telles que IJ France, SIMV, CIDJ, PSE, Mairie de Champigny et IJBOX.}
      \item {Participation à des réunions clients afin de clarifier les besoins fonctionnels et de proposer des solutions techniques adaptées.}
    \end{cvitems}
  }

\vspace{12pt}

\cventry
  {Stagiaire Ingénieur Logiciel}
  {\textcolor{awesome}{Lear Corporation}}
  {Kénitra, Maroc}
  {Fév. 2025 -- Juil. 2025}
  {
    \begin{cvitems}
      \item {Conception et développement d’une plateforme web interne pour la digitalisation des processus de recrutement, de formation et de certification, en remplacement de fichiers Excel et de procédures papier.}
      \item {Mise en place de modules d’évaluation automatisés et de mécanismes de notification pour le suivi des stagiaires et des certifications.}
      \item {Traitement et transformation des logs de production en dashboards opérationnels utilisés par les équipes terrain.}
      \item {Développement d’un prototype de détection de défauts basé sur la vision par ordinateur pour le contrôle qualité en environnement \textbf{Just-In-Time}.}
    \end{cvitems}
  }

\vspace{12pt}

\cventry
  {Développeur Logiciel Freelance}
  {\textcolor{awesome}{Twins Groupe}}
  {Fès, Maroc}
  {Jan. 2025 -- Fév. 2025}
  {
    \begin{cvitems}
      \item {Développement de fonctionnalités backend et de templates frontend en \textbf{PHP/Symfony} pour des plateformes institutionnelles et du secteur public.}
      \item {Livraison d’une solution prête pour la production, conforme aux exigences fonctionnelles et techniques du client.}
    \end{cvitems}
  }

\end{cventries}

%-------------------------------------------------------------------------------
% FORMATION
%-------------------------------------------------------------------------------
\cvsection{Formation}

\begin{cventries}
  \cventry
    {Diplôme d’Ingénieur d’État en Informatique — SIAD}
    {ENSA Tétouan}
    {Tétouan, Maroc}
    {2020 -- 2025}
    {
      \begin{cvitems}
        \item {\textbf{Spécialisation :} Systèmes d’Information et Aide à la Décision}
        \item {\textbf{Projet de fin d’études :} Détection de défauts par vision par ordinateur pour le contrôle qualité en milieu industriel.}
      \end{cvitems}
    }
\end{cventries}
\vspace{4pt}

%-------------------------------------------------------------------------------
% COMPÉTENCES
%-------------------------------------------------------------------------------
\cvsection{Compétences Techniques}

\begin{cvskills}
  \cvskill
    {Backend \& APIs}
    {Python (Django), Java (Spring Boot), PHP (Symfony), REST APIs, WebSockets}

  \cvskill
    {Bases de données}
    {PostgreSQL, MySQL, optimisation SQL, indexation, migrations, performance des requêtes}

  \cvskill
    {Web}
    {React, JavaScript, HTML, CSS, responsive design}

  \cvskill
    {Données \& IA}
    {Concepts ETL, BigQuery, traitement de données opérationnelles, vision par ordinateur, YOLO, OpenCV}

  \cvskill
    {DevOps \& Outils}
    {Git, Docker, Linux, CI/CD, Redis, Nginx, debugging et profiling}
\end{cvskills}

%-------------------------------------------------------------------------------
% LANGUES
%-------------------------------------------------------------------------------
\cvsection{Langues}

\begin{cvparagraph}
\textbf{Arabe} : langue maternelle \quad\textbar\quad
\textbf{Français} : courant (C1) \quad\textbar\quad
\textbf{Anglais} : courant (C1)
\end{cvparagraph}

\end{document}
